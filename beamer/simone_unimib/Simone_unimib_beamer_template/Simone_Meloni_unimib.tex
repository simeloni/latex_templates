\documentclass[9pt, xcolor=dvipsnames]{beamer}
\usetheme{Boadilla}
\usepackage[utf8]{inputenc}
\usepackage[english]{babel}
\usepackage{amsmath}
\usepackage{amsfonts}
\usepackage{amssymb}
\usepackage{eso-pic}
\usepackage{framed}
\usepackage{fancyvrb}
\usepackage{graphicx}
\usepackage{color}


%===============LINE FOR TITLES=================
\usepackage{tikz}
\usetikzlibrary{shapes.arrows, fadings}
\newcommand{\topline}{%
	\tikzfading[name=fade left, right color = transparent!0, left color=transparent!100]
	\tikz[remember picture, overlay] {%
	\fill[Periwinkle, path fading=fade left] ([yshift=-1cm]current page.north west) rectangle ([yshift=-1.15cm, xshift=\paperwidth]current page.north west);
			% -- ([yshift=-1cm, xshift=\paperwidth]current %page.north west);}}
			}}
\newcommand{\backupbegin}{
  \newcounter{finalframe}
   \setcounter{finalframe}{\value{framenumber}}
}
\newcommand{\backupend}{
   \setcounter{framenumber}{\value{finalframe}}
}

%===============ENUMERATION STYLE===============
\setbeamertemplate{itemize item}[circle]


%================COLORS=========================

\definecolor{SimoneWarmGreyDark}{rgb}{0.274509804,0.254901961, 0.235294118}

\setbeamercolor{normal text}{fg=SimoneWarmGreyDark}

%====================TITLE PAGE=================
\def\titlepage{\usebeamertemplate{titlepage}}
\setbeamertemplate{titlepage}
{
	% Add background to title page
	\AddToShipoutPictureFG*{\includegraphics[width=\paperwidth]{images/Sigillo_background.png}}

	\begin{minipage}[t][\paperheight]{\textwidth}
		\begin{minipage}[b][0.15\paperheight]{0.4\textwidth}
	\vspace*{10mm}
	\begin{columns}
	\begin{column}{0.3\textwidth}
	\includegraphics[height=14mm]{images/Sigillo.jpg}\par
	\end{column}
	\begin{column}{0.5\textwidth}
	\ifx\insertinstitut\@empty%
	\else%
		\vspace*{3mm}
		{\usebeamerfont{institute}\usebeamercolor[fg]{institute}\insertinstitute\par}%
	\fi% 
	\end{column}
	\end{columns}
	\end{minipage}
	
	\vspace*{5mm}
	\ifx\insertsubtitle\@empty%
	\else%
		{\usebeamerfont{title}\usebeamercolor[fg]{title}\MakeUppercase{\inserttitle}\par}%
	\fi%
	\ifx\insertsubtitle\@empty%
	\else%
		{\usebeamerfont{subtitle}\usebeamercolor[fg]{subtitle}\insertsubtitle\par}%
		\vspace*{5mm}
	\fi%
	\ifx\insertdate\@empty%
	\else%
		{\usebeamerfont{date}\usebeamercolor[fg]{date}\insertdate\par}%
	\fi% 
	\vspace*{10mm}
	\begin{tabular}{cc}
	\noindent
	\textbf{Supervisors:} & Mark Whitehead\\
						  & Conor Fitzpatrick
	\end{tabular}
	\vfill
	
	\ifx\insertauthor\@empty%
	\else%
		{\usebeamerfont{author}\usebeamercolor[fg]{author}\insertauthor\par}%
	\fi%
	s.meloni1@campus.unimib.it
	\vspace*{5mm}
	
\end{minipage}
}


\title{Titolo}
\subtitle{Sottotitolo}
\author{\textbf{Simone Meloni}}
\institute{\par \textbf{Milano Bicocca} University}
\date{August 30, 2016}







\begin{document}

\begin{frame}[plain]
\titlepage
\end{frame}

%\begin{frame}
%\tableofcontents
%\end{frame}

\begin{frame}{About Myself}
\topline

\end{frame}

\begin{frame}{Trigger Upgrade}
\topline
	\begin{columns}
	\begin{column}{0.5\textwidth}
		\begin{centering}
	     \includegraphics[scale = 0.45]{images/Upgraded_trigger.png}
		\end{centering}
	\end{column}
	\begin{column}{0.5\textwidth}
		\begin{itemize}
	 	\item Run3 configuration : 
	 		\begin{itemize}
	 		 \item L = $2 x 10^{33}cm^{-2} s^{-1} $ 
	 		 \item Online Calibration
	 	 	 \item Full Software Trigger
	 		\end{itemize}
		\end{itemize}
		
		\begin{itemize}
	  	 \item The Full Software trigger will process the full inelastic collision rate.
		\end{itemize}
		
		\begin{itemize}
			\item \textbf{Challenge:} How do we distribute the limited bandwidth to disk across the LHCb physics programme?		
			\end{itemize}
	\end{column}
	
	
	\end{columns}
\end{frame}

\begin{frame}{Bandwidth division in the charm sector}
\topline
\begin{itemize}
\item \textbf{My work}: Develop a framework to allow trigger selections to be adjusted in order to distribute the available bandwidth

\begin{itemize}
 \item Study of the physics potential as a function of the output bandwidth.
\end{itemize}
\end{itemize}

\begin{itemize}
 \item Focus on the \textbf{charm sector}, and in particular:
 \begin{columns}
			\begin{column}{0.5 \textwidth}
			\begin{center}
			
			\begin{itemize}
			 \item[] $D^{+}\rightarrow K^+K^-\pi^{+}$
			 \item[] $D^{0}\rightarrow K^+K^-$
			 \item[] $D^{0}\rightarrow K^+K^-\pi^{+}\pi^{-}$
			 \item[] $D^{0}\rightarrow K^0_S\pi^{+}\pi^{-}$
			\end{itemize}
			\end{center}
			\end{column}
			
			\begin{column}{0.5 \textwidth}
			\item[] where the $D^0$ comes from $D^{+*}\rightarrow D^{0} \pi^{+}$
			\end{column}
			\end{columns}
\end{itemize}

\begin{itemize}
\item Using latest Upgrade MC signal and minbias samples with the upgrade reconstruction sequence
\begin{itemize}
	\item A big thank to Mark Whitehead and Agnieszka Dziurda
\end{itemize}
\end{itemize}
\end{frame}

\begin{frame}{Idea}
\topline
	\begin{itemize}
	\item Single dial for each channel (or analysis) that tunes the bandwidth used!
	\end{itemize}
	
	\begin{itemize}
	\item \textbf{For example}$\longrightarrow$ output of a continuous classifier 
	\end{itemize}
	
	\begin{itemize}
	\item First part of the project: Train an MVA for each channel
	\begin{itemize}
		\item I have written a \texttt{Python} code that performs binary classification
	\end{itemize}
	
	\begin{itemize}
		\item Analysts can use this code to train their own MVA 
	\end{itemize}
	\end{itemize}

	\begin{columns}
	\begin{column}{0.7\textwidth}
	\begin{framed}
	\begin{figure}
	\centering
	\includegraphics[scale=0.3]{images/SMVA_docu.png}
	
	\end{figure}
	\end{framed}
	\end{column}
	
	\begin{column}{0.3\textwidth}
	\begin{figure}
	\includegraphics[scale=0.2]{images/scikit-learn.png}
	\end{figure}
	\begin{center}
	 Code and Documentation available at \texttt{https://gitlab.com/\\simeloni/SMVA}
	\end{center}
	\end{column}
	\end{columns}
	
\end{frame}

\begin{frame}{SMVA}
\topline
\begin{itemize}
\item Performs binary classification (i.e. signal vs background separation)
\end{itemize}

\begin{itemize}
\item Produces \textit{TMVA-Like} Plots in an automized way. Simple to use:
\end{itemize}

\begin{columns}
\begin{column}{0.40\textwidth}
\begin{center}
\begin{framed}
\begin{framed}
\texttt{inputfile.py}
\end{framed}
\begin{itemize}
	\item file paths
	\item Preselections
	\item MVA and training var's
\end{itemize}
\end{framed}
\end{center}
\end{column}

\begin{column}{0.20\textwidth}
%\begin{columns}
%\begin{column}{0.33\textwidth}
%\begin{center}
%$\longrightarrow$
%\end{center}
%\end{column}

%\begin{column}{0.33\textwidth}
%\begin{figure}
%\begin{flushleft}
%\includegraphics[scale=0.23]{images/Brain_Train.jpeg}
%\end{flushleft}
%\end{figure}
%\end{column}

%\begin{column}{0.33\textwidth}
%\begin{center}
%$\longrightarrow$
%\end{center}
%\end{column}
%\end{columns}

\begin{figure}
\begin{flushleft}
\includegraphics[scale=0.23]{images/Brain_Train.jpeg}
\end{flushleft}
\end{figure}

\begin{center}
$\Longrightarrow$
\end{center}
\end{column}

\begin{column}{0.40\textwidth}
\begin{figure}
\includegraphics[scale=0.19]{images/MVA_root_output.png}
\end{figure}
\end{column}
\end{columns}

\begin{itemize}
\item Options available to enable specific Plots and features
\end{itemize}
\end{frame}

\begin{frame}
\begin{minipage}{1.0\textwidth}
	
	
	\begin{minipage}{0.47\textwidth}
	\begin{framed}
		\tiny\texttt{--MakeROC}
		%\begin{figure}
		\includegraphics[scale=0.22]{images/ROC_curve_only_AdaBoost.pdf}
		%\end{figure}
	\end{framed}
	\end{minipage}	
	\begin{minipage}{0.47\textwidth}
	\begin{framed}	
		\tiny\texttt{--CompareTrainTest}
		%\begin{figure}
		\includegraphics[scale=0.22]{images/overtraining_check.pdf}
		%\end{figure}
	\end{framed}
	\end{minipage}
	
	\begin{minipage}{0.47\textwidth}
	\begin{framed}
		
		\tiny\texttt{--ComparePerformances}
		\includegraphics[scale=0.22]{images/MVA_comparison.pdf}
	\end{framed}
	\end{minipage}	
	\begin{minipage}{0.47\textwidth}
	\begin{framed}
		\tiny\texttt{--MakePlots}
		%\begin{figure}
		\includegraphics[scale=0.22]{images/eff_vs_ret.pdf}
		%\end{figure}
	\end{framed}
	\end{minipage}
	
\end{minipage}




\end{frame}




\begin{frame}{Bandwidth division algorithm}
\topline
\begin{itemize}
\item \textbf{Idea:} use the single dial to find the right MVA selection for all the channels to \textcolor{blue}{fit into the Bandwidth limit} \textcolor{red}{maximizing the total signal efficiency}
\end{itemize}

\begin{itemize}
	\item \textbf{Algorithm:} minimize this function...
\end{itemize}
\begin{columns}
\begin{column}{0.5\textwidth}
\begin{equation*}
		\chi^2 = \sum_{channels} \left(1-\dfrac{\textcolor{red}{\epsilon_s}}{\textcolor{red}{\epsilon_{max}}}\right)^2
\end{equation*}
\end{column}
\begin{column}{0.5\textwidth}
\begin{equation*}
		\epsilon_s = \dfrac{N(truth, presel., MVAsel.)}{N(truth)}
		\end{equation*}
\end{column}
\end{columns}

\begin{itemize}
  	\item[] $\epsilon_{max}$ $=$ maximum efficiency affordable for a single channel providing it with the maximum bandwidth.
   \end{itemize}

\begin{center}
...multiplying the efficiency by a penalty factor whenever $BW > BW_{limit}$
\end{center}

\begin{center}
		\textcolor{blue}{penalty} $ = \dfrac{BW_{limit}}{BW}$ 
	\end{center}


\end{frame}

\begin{frame}{Bandwidth division algorithm}
\topline
\begin{itemize}
 \item \textbf{first iteration:} For each channel independently minimize this function
\end{itemize}

	\begin{equation*}
		\tilde{\chi}^2 = (1-\epsilon_{max})^2
	\end{equation*}

	the cut found by the minimizer gives you $\epsilon_{max}$
	
\begin{itemize}
\item \textbf{second iteration:} use the $\epsilon_{max}$ found to construct the $\chi^2$ and minimize it.
\end{itemize}

	\begin{equation*}
		\chi^2 = \sum_{channels} \left(1-\dfrac{\epsilon}{\epsilon_{max}}\right)^2
	\end{equation*}
	
\begin{itemize}
	\item \textbf{OUTPUT:} A set of cuts with which we can evaluate the signal efficiencies, output Bandwidth and Rate for each of the channels
\end{itemize}

\end{frame}

\begin{frame}{Bandwidth division output}
\topline
\begin{itemize}
\item Example of an output graph for the bandwidth division tool
\end{itemize}

\begin{columns}
\begin{column}{0.5\textwidth}
\begin{itemize}
	\item[]{\includegraphics[scale=0.2]{images/bwdiv_legend_red.png}} At the first iteration, minimizing
\end{itemize}
\begin{small}
\begin{equation*}
		\tilde{\chi}^2 = \left(1-\epsilon_{max}\right)^2
\end{equation*}
\end{small}
\end{column}

\begin{column}{0.5\textwidth}
\begin{itemize}
	\item[]{\includegraphics[scale=0.2]{images/bwdiv_legend_blu.png}} At the second iteration, minimizing
\end{itemize}
\begin{small}
\begin{equation*}
		\chi^2 = \sum_{channels} \left(1-\dfrac{\epsilon_s}{\epsilon_{max}}\right)^2
\end{equation*}
\end{small}
\end{column}
\end{columns}

\begin{columns}
\begin{column}{0.5\textwidth}
\begin{figure}
\includegraphics[scale=0.33]{images/60MB_NO_TURBO/MinbiasBandwidth.eps}
\end{figure}
\end{column}

\begin{column}{0.5\textwidth}
\begin{itemize}
\item Assuming a  \textbf{Bandwidth limit of 60 MB/s} and an event size for each of the channels: 
\end{itemize}
\begin{center}
\begin{tabular}{|c|c|}
\hline 
& \small event Size [kB/evt]\\
\hline
\small$D^{+}\rightarrow K^+K^-\pi^{+}$ & 14\\
\small$D^{0}\rightarrow K^+K^-$ & 12\\
\small$D^{0}\rightarrow K^+K^-\pi^{+}\pi^{-}$ & 14 \\
\small$D^{0}\rightarrow K^0_S\pi^{+}\pi^{-}$ & 70\\
\hline
\end{tabular}
\end{center}

\end{column}
\end{columns}
\end{frame}

\begin{frame}{Event Size in the bandwidth division}
\topline
\begin{itemize}
\item The way in which the events are saved affects the bandwidth division:
\end{itemize}

\begin{center}
\begin{tabular}{|cccc|}
\hline 
$D^{+}\to K^+K^-\pi^{+}$ & $D^{+}\to K^+K^-\pi^{+}$&
$D^{0}\to K^+K^-$ &
$D^{0}\to K^0_S\pi^{+}\pi^{-}$\\
\hline
&&&\\
14 kB/evt. & 12 kB/evt. & 14 kB/evt. & \textcolor{red}{70 kB/evt.}\\
&&&\\
\hline
\end{tabular}
\end{center}

\begin{columns}
\begin{column}{0.5\textwidth}
\begin{figure}
\includegraphics[scale=0.32]{images/60MB_NO_TURBO/MinbiasRate.eps}
\end{figure}
\end{column}

\begin{column}{0.5\textwidth}
\begin{itemize}
		\item The bigger the event size (for a given bandwidth), the harder the MVA cut
		\begin{itemize}
		\item Channels with bigger event size get less rate because the MVA has to cut harder in order to fit them inside the Bandwith limit
		\end{itemize}
\end{itemize}
\end{column}
\end{columns}



\end{frame}

\begin{frame}{Event Size in the bandwidth division}
\topline
\begin{itemize}
\item The way in which the events are saved affects the bandwidth division:
\end{itemize}

\begin{center}
\begin{tabular}{|cccc|}
\hline 
$D^{+}\to K^+K^-\pi^{+}$ & $D^{+}\to K^+K^-\pi^{+}$&
$D^{0}\to K^+K^-$ &
$D^{0}\to K^0_S\pi^{+}\pi^{-}$\\
\hline
&&&\\
14 kB/evt. & 12 kB/evt. & 14 kB/evt. & \textcolor{red}{14 kB/evt.}\\
&&&\\
\hline
\end{tabular}
\end{center}

\begin{columns}
\begin{column}{0.5\textwidth}
\begin{figure}
\includegraphics[scale=0.32]{images/60MB_TURBO/MinbiasRate.eps}
\end{figure}
\end{column}

\begin{column}{0.5\textwidth}
\begin{itemize}
		\item The bigger the event size (for a given bandwidth), the harder the MVA cut
		\begin{itemize}
		\item Channels with bigger event size get less rate because the MVA has to cut harder in order to fit them inside the Bandwith limit
		\end{itemize}
\end{itemize}
\end{column}
\end{columns}



\end{frame}

\begin{frame}{Bandwidth Scans}
\topline
\begin{itemize}
\item How does the division vary with the available bandwidth?
\item Bandwidth scans can be obtained running the same algorithm for different limits scenarios
\end{itemize}

\begin{columns}
\begin{column}{0.5\textwidth}
\begin{center}
\small\textbf{$D^{0}\to K^0_S\pi^+\pi^-$}(70 kB/evt.)
\end{center}
\begin{figure}
\includegraphics[scale=0.3]{images/projection_plots/NO_TURBO/Signal_Eff.eps}
\end{figure}
\end{column}


\begin{column}{0.5\textwidth}
\begin{center}
\small\textbf{$D^{0}\to K^0_S\pi^+\pi^-$}(14 kB/evt.) 
\end{center}
\begin{figure}
\includegraphics[scale=0.3]{images/projection_plots/TURBO/Signal_Eff.eps}
\end{figure}
\end{column}
\end{columns}

\begin{itemize}
\item Big enhancement in the Signal Efficiency for the line that turned to Turbo Stream
\end{itemize}

\begin{itemize}
\item If we turn one of the lines from Turbo to Non-Turbo stream, the other lines aren't badly affected!
\end{itemize}

\end{frame}

\begin{frame}{Conclusions}
\topline

\begin{center}
 With the tools I have developed analysts can provide HLT2 selections that can be adjusted in a controlled manner to fit within the upgrade Bandwidth. The tools consist of two stages:
\end{center}

\begin{itemize}
	\item \textbf{First stage:} MVA training using \texttt{Scikit Learn}
	\begin{itemize}
		\item Developed an agile \texttt{Python} script to automate the production of \textit{TMVA-like} plots
	\end{itemize}
\end{itemize}

\begin{itemize}
\item \textbf{Second stage:} Used the MVA output to perform the bandwidth division in the charm sector with the upgraded conditions
\begin{itemize}
	\item Implemented a quantitative way to divide the bandwidth amongst the physics channels
	\item Performed some Projections in different output bandwidth scenarios
\end{itemize}
\end{itemize}

\begin{itemize}
\item \textbf{Next goal:} use this tool to extraploate to the entire charm programme and outline the expected bandwidth available for each channel in the upgrade conditions.
\end{itemize}

\end{frame}

\begin{frame}

\begin{center}
	\textbf{THANK YOU FOR YOUR ATTENTION!}
\end{center}
\end{frame}

\backupbegin
\begin{frame}
\begin{center}
BACKUP
\end{center}
\end{frame}

\begin{frame}{Bandwidth division}
\topline
\begin{itemize}
	\item The important plot for the bandwidth division tool is the signal efficiency versus the Output Bandwidth
\end{itemize}

\begin{columns}
	\begin{column}{0.5\textwidth}
		\begin{equation*}
		\epsilon_s = \dfrac{N(truth, presel., MVAsel.)}{N(truth)}
		\end{equation*}
		 
		\begin{equation*}
		Bandwidth = \epsilon_{minbias} \times R_{minbias} \times Evt. Size
\end{equation*}
		
	\end{column}
	\begin{column}{0.5\textwidth}
	\begin{figure}
		\includegraphics[scale=0.25]{images/eff_vs_ret.pdf}
	\end{figure}
	\end{column}
\end{columns}

\begin{itemize}
	\item Preselection = (Loose) HLT2 selections (Thanks to Mark Williams)
\end{itemize}

\begin{center}
\begin{tabular}{|c|cccc|}
\hline 
&\textbf{D2KKpi} & \textbf{D2KK} &\textbf{D2KKpipi} & \textbf{D2Kspipi} \\
\hline 
Event Size  &14 kB& 12 kB&14 kB& 70kB\\
Signal Eff.\footnote{wrt truth matched cand.} &44.1(4)$\%$ &63.33(5)$\%$&76.6(8)$\%$&68.7(1)$\%$\\


Background Eff.\footnote{wrt anti-truth matched cand.}&0.0656(4)$\%$ &0.279(3)$\%$ &1.690(6)$\%$&3.621(9)$\%$\\

\hline 
\end{tabular} 
\end{center}
\end{frame}

\begin{frame}{Event Size in the bandwidth division}
\topline
\begin{itemize}
\item The way in which the events are saved affects the bandwidth division:
	\begin{itemize}
		\item The bigger the event size, the harder the MVA cut
	\end{itemize}
\end{itemize}

\begin{center}
\begin{tabular}{|cccc|}
\hline 
$D^{+}\to K^+K^-\pi^{+}$ & $D^{+}\to K^+K^-\pi^{+}$&
$D^{0}\to K^+K^-$ &
$D^{0}\to K^0_S\pi^{+}\pi^{-}$\\
\hline
&&&\\
14 kB/evt. & 12 kB/evt. & 14 kB/evt. & 70 kB/evt.\\
&&&\\
\hline
\end{tabular}
\end{center}

\begin{columns}
\begin{column}{0.5\textwidth}
\begin{figure}
\includegraphics[scale=0.32]{images/60MB_NO_TURBO/MinbiasRate.eps}
\end{figure}
\end{column}

\begin{column}{0.5\textwidth}
\begin{figure}
\includegraphics[scale=0.34]{images/60MB_NO_TURBO/SignalEfficiencies.eps}
\end{figure}
\end{column}
\end{columns}



\end{frame}

\begin{frame}{Event Size in the bandwidth division}
\topline
\begin{itemize}
\item The way in which the events are saved affects the bandwidth division:
	\begin{itemize}
		\item The bigger the event size, the harder the MVA cut
	\end{itemize}
\end{itemize}

\begin{center}
\begin{tabular}{|cccc|}
\hline 
$D^{+}\to K^+K^-\pi^{+}$ & $D^{+}\to K^+K^-\pi^{+}$&
$D^{0}\to K^+K^-$ &
$D^{0}\to K^0_S\pi^{+}\pi^{-}$\\
\hline
&&&\\
14 kB/evt. & 12 kB/evt. & 14 kB/evt. & 14 kB/evt.\\
&&&\\
\hline
\end{tabular}
\end{center}

\begin{columns}
\begin{column}{0.5\textwidth}
\begin{figure}
\includegraphics[scale=0.32]{images/60MB_TURBO/MinbiasRate.eps}
\end{figure}
\end{column}

\begin{column}{0.5\textwidth}
\begin{figure}
\includegraphics[scale=0.34]{images/60MB_TURBO/SignalEfficiencies.eps}
\end{figure}
\end{column}
\end{columns}



\end{frame}

\begin{frame}{Preselections}
\topline
\begin{itemize}
\item[]{}
\item Kinematical Cuts, Track quality, Vertex quality, Ghost Probability, and a loose Mass Window on the candidates
\end{itemize}

\begin{center}
\begin{tabular}{|c|cccc|}
\hline 
&\textbf{D2KKpi}& \textbf{D2KK} &\textbf{D2KKpipi} & \textbf{D2Kspipi} \\
\hline 
\tiny Processed events  & 92704&103163&93581&102468\\
\tiny Reco. Candidates &2953517&231261&406361&345262\\
\hline
\tiny Truth matched cand. &13255&11130&2299&1174\\
\tiny Truth matched and preselected cand. &5887&7049&1762&806\\
\hline
\tiny Signal Efficiency \footnote{wrt truth matched cand.}&44.1(4)$\%$ &63.33(5)$\%$&76.6(8)$\%$&68.7(1)$\%$\\
\tiny Background Efficiency \footnote{wrt anti-truth matched cand.}&0.0656(4)$\%$ &0.279(3)$\%$ &1.690(6)$\%$&3.621(9)$\%$\\
\hline

\end{tabular} 

\begin{itemize}
\item Big difference on the number of reco. candidates with respect to the number of events in the D2KKpi channel: there is no $\Delta_m$ cut (No $D^{*}$ decay)
\end{itemize}
\end{center}

\end{frame}

\begin{frame}{PreSelection on the Minbias events}
\topline
\begin{center}
\begin{tabular}{|c|cccc|}
\hline 
&\textbf{D2KKpi}& \textbf{D2KK} &\textbf{D2KKpipi} & \textbf{D2Kspipi}\\
\hline 
\tiny Processed events  &2984550&2963306&2954144&2963773\\
\tiny Reco. Candidates &45044857&2927035&4709605&4085688\\
\hline
\tiny Anti-Truth matched cand. &45039035&2926918&4709569&4085532\\
\tiny Anti-Truth matched and preselected cand. &29552&8190&79581&147950\\
\hline
\hline
\end{tabular} 
\end{center}
\end{frame}

\begin{frame}[fragile]{Preselections}
\topline
\begin{itemize}
\item \textbf{D2KKpi}
\end{itemize}

\begin{verbatim}
[				   '(cand_M > 1700)',
				   '(cand_M < 2100)',
				   '((dau1_PT + dau2_PT + dau3_PT) > 3000)', 
				   '(TMath::ACos(cand_LOKI_BPVDIRA) < 0.010)',
				   '(cand_LOKI_BPVLTIME > 0.0004)',
				   '(dau1_PT > 250)',
				   '(dau3_PT > 250)',
				   '(dau1_TRACK_CHI2NDOF < 6.0)',
				   '(dau2_TRACK_CHI2NDOF < 6.0)',
				   '(dau3_TRACK_CHI2NDOF < 6.0)',
				   '(dau1_LOKI_MIPCHI2DV_PRIMARY > 4.0)',
				   '(dau2_LOKI_MIPCHI2DV_PRIMARY > 4.0)',
				   '(dau1_TRACK_GhostProb<0.4)', 
				   '(dau2_TRACK_GhostProb<0.4)',
				   '(dau3_TRACK_GhostProb<0.4)',
				   '(cand_LOKI_VFASPF_VCHI2VDOF < 6.)']
\end{verbatim}
\end{frame}

\begin{frame}[fragile]{Preselections}
\topline
\begin{itemize}
\item \textbf{D2KK}
\end{itemize}

\begin{verbatim}
						   ['(TMath::ACos(dau1_LOKI_BPVDIRA) < 0.0173)', 
						   '(dau1_LOKI_BPVVDCHI2 > 25.0)',
						   '(dau1_PT > 1000.)',
						   '(dau1_LOKI_VFASPF_VCHI2VDOF < 10.)',
						   '(Ddau1_LOKI_MIPCHI2DV_PRIMARY > 4.0)',
						   '(Ddau2_LOKI_MIPCHI2DV_PRIMARY > 4.0)',
 						   '(Ddau1_PT > 500.)',
						   '(Ddau2_PT > 500.)',
						   '(Ddau1_P  > 5000.)', 
						   '(Ddau2_P  > 5000.)', 
						   '(Ddau1_TRACK_CHI2NDOF < 3.0)',
						   '(Ddau2_TRACK_CHI2NDOF < 3.0)',
					           '(Ddau1_TRACK_GhostProb<0.4)', 
						   '(Ddau2_TRACK_GhostProb<0.4)',
						   '(dau1_LOKI_VFASPF_VCHI2VDOF < 10.0)']
\end{verbatim}
\end{frame}

\begin{frame}[fragile]{Preselections}
\topline
\begin{itemize}
\item \textbf{D2KKpi}
\end{itemize}

\begin{verbatim}
[				   '(cand_M > 1700)',
				   '(cand_M < 2100)',
				   '((dau1_PT + dau2_PT + dau3_PT) > 3000)', 
				   '(TMath::ACos(cand_LOKI_BPVDIRA) < 0.010)',
				   '(cand_LOKI_BPVLTIME > 0.0004)',
				   '(dau1_PT > 250)',
				   '(dau3_PT > 250)',
				   '(dau1_TRACK_CHI2NDOF < 6.0)',
				   '(dau2_TRACK_CHI2NDOF < 6.0)',
				   '(dau3_TRACK_CHI2NDOF < 6.0)',
				   '(dau1_LOKI_MIPCHI2DV_PRIMARY > 4.0)',
				   '(dau2_LOKI_MIPCHI2DV_PRIMARY > 4.0)',
				   '(dau1_TRACK_GhostProb<0.4)', 
				   '(dau2_TRACK_GhostProb<0.4)',
				   '(dau3_TRACK_GhostProb<0.4)',
				   '(cand_LOKI_VFASPF_VCHI2VDOF < 6.)']
\end{verbatim}
\end{frame}

\begin{frame}[fragile]{Preselections}
\topline
\begin{itemize}
\item \textbf{D2KKpipi}
\end{itemize}

\begin{Verbatim}[fontsize=\small]
				['(Ddau1_PT > 250.)',
						   '(Ddau2_PT > 250.)',
						   '(Ddau3_PT > 250.)',
						   '(Ddau4_PT > 250.)',
	            '((Ddau1_PT + Ddau2_PT + Ddau3_PT + Ddau4_PT ) > 1800.)',
						   '(Ddau1_LOKI_MIPCHI2DV_PRIMARY > 3)',
						   '(Ddau2_LOKI_MIPCHI2DV_PRIMARY > 3)',
						   '(Ddau3_LOKI_MIPCHI2DV_PRIMARY > 3)',
		                                   '(Ddau4_LOKI_MIPCHI2DV_PRIMARY > 3)',
						   '(Ddau1_TRACK_GhostProb <0.4)',
						   '(Ddau2_TRACK_GhostProb <0.4)',
						   '(Ddau3_TRACK_GhostProb <0.4)',
						   '(Ddau4_TRACK_GhostProb <0.4)',
						   '(dau1_M > 1700.)',
						   '(dau1_M < 2100.)',
						   '(dau1_LOKI_VFASPF_VCHI2VDOF < 12.0)',
						  '(TMath::ACos(dau1_LOKI_BPVDIRA < 0.20))',
						   '(dau1_LOKI_BPVLTIME > 0.0001)',
						   '(dau1_PT > 2000.)',
						   '(dau1_LOKI_BPVVDCHI2 > 25.)']
	
\end{Verbatim}
\end{frame}

\begin{frame}[fragile]{Preselections}
\topline
\begin{itemize}
\item \textbf{D2KKpi}
\end{itemize}

\begin{verbatim}
[				   '(cand_M > 1700)',
				   '(cand_M < 2100)',
				   '((dau1_PT + dau2_PT + dau3_PT) > 3000)', 
				   '(TMath::ACos(cand_LOKI_BPVDIRA) < 0.010)',
				   '(cand_LOKI_BPVLTIME > 0.0004)',
				   '(dau1_PT > 250)',
				   '(dau3_PT > 250)',
				   '(dau1_TRACK_CHI2NDOF < 6.0)',
				   '(dau2_TRACK_CHI2NDOF < 6.0)',
				   '(dau3_TRACK_CHI2NDOF < 6.0)',
				   '(dau1_LOKI_MIPCHI2DV_PRIMARY > 4.0)',
				   '(dau2_LOKI_MIPCHI2DV_PRIMARY > 4.0)',
				   '(dau1_TRACK_GhostProb<0.4)', 
				   '(dau2_TRACK_GhostProb<0.4)',
				   '(dau3_TRACK_GhostProb<0.4)',
				   '(cand_LOKI_VFASPF_VCHI2VDOF < 6.)']
\end{verbatim}
\end{frame}

\begin{frame}[fragile]{Preselections}
\begin{itemize}
\item \textbf{D2Kspipi}
\end{itemize}

\begin{verbatim}
		['(dau1_M > 1700)',
						   '(dau1_M < 2100)',
						   '((Ddau1_PT + Ddau2_PT + Ddau3_PT) > 1500)',
						   '(dau1_PT > 1800. )',
						   '(TMath::ACos(dau1_LOKI_BPVDIRA) < 0.0346)',
						   '(dau1_LOKI_BPVVDCHI2 > 20.0)',
						   '(dau1_LOKI_VFASPF_VCHI2VDOF < 20.)',
						   '(dau1_LOKI_BPVLTIME > 0.0001 )', #ns
						   '(Ddau1_PT > 250.)',
						   '(Ddau2_PT > 250.)',
						   '(Ddau3_PT > 250.)',
						   '(Ddau1_LOKI_MIPCHI2DV_PRIMARY > 3.)',
					           '(Ddau2_LOKI_MIPCHI2DV_PRIMARY > 3.)',
						   '(Ddau3_LOKI_MIPCHI2DV_PRIMARY > 3.)',
						   '(Ddau2_TRACK_GhostProb < 0.4)',
						   '(Ddau3_TRACK_GhostProb < 0.4)']
				  
\end{verbatim}
\end{frame}

\begin{frame}{Projection Plots: Bandwidth}
\topline
\begin{columns}
\begin{column}{0.5\textwidth}
\begin{figure}
\caption{D2KSpipi NON Turbo}
\includegraphics[scale=0.3]{images/projection_plots/NO_TURBO/Bandwidth.eps}
\end{figure}
\end{column}
\begin{column}{0.5\textwidth}
\begin{figure}
\caption{D2KSpipi Turbo}
\includegraphics[scale=0.3]{images/projection_plots/TURBO/Bandwidth.eps}
\end{figure}
\end{column}
\end{columns}
\end{frame}

\begin{frame}{Projection Plots: max Bandwidth}
\topline
\begin{columns}
\begin{column}{0.5\textwidth}
\begin{figure}
\caption{D2KSpipi NON Turbo}
\includegraphics[scale=0.3]{images/projection_plots/NO_TURBO/max_Bandwidth.eps}
\end{figure}
\end{column}
\begin{column}{0.5\textwidth}
\begin{figure}
\caption{D2KSpipi Turbo}
\includegraphics[scale=0.3]{images/projection_plots/TURBO/max_Bandwidth.eps}
\end{figure}
\end{column}
\end{columns}
\end{frame}

\begin{frame}{Projection Plots: Retention}
\topline
\begin{columns}
\begin{column}{0.5\textwidth}
\begin{figure}
\caption{D2KSpipi NON Turbo}
\includegraphics[scale=0.3]{images/projection_plots/NO_TURBO/Minbias_Retention.eps}
\end{figure}
\end{column}
\begin{column}{0.5\textwidth}
\begin{figure}
\caption{D2KSpipi Turbo}
\includegraphics[scale=0.3]{images/projection_plots/TURBO/Minbias_Retention.eps}
\end{figure}
\end{column}
\end{columns}
\end{frame}

\begin{frame}{Projection Plots: max Retention}
\topline
\begin{columns}
\begin{column}{0.5\textwidth}
\begin{figure}
\caption{D2KSpipi NON Turbo}
\includegraphics[scale=0.3]{images/projection_plots/NO_TURBO/max_Minbias_Retention.eps}
\end{figure}
\end{column}
\begin{column}{0.5\textwidth}
\begin{figure}
\caption{D2KSpipi Turbo}
\includegraphics[scale=0.3]{images/projection_plots/TURBO/max_Minbias_Retention.eps}
\end{figure}
\end{column}
\end{columns}
\end{frame}

\begin{frame}{Projection Plots: max Rate}
\topline
\begin{columns}
\begin{column}{0.5\textwidth}
\begin{figure}
\caption{D2KSpipi NON Turbo}
\includegraphics[scale=0.3]{images/projection_plots/NO_TURBO/max_Rate.eps}
\end{figure}
\end{column}
\begin{column}{0.5\textwidth}
\begin{figure}
\caption{D2KSpipi Turbo}
\includegraphics[scale=0.3]{images/projection_plots/TURBO/max_Rate.eps}
\end{figure}
\end{column}
\end{columns}
\end{frame}

\begin{frame}{Projection Plots: max Signal Efficiency}
\topline
\begin{columns}
\begin{column}{0.5\textwidth}
\begin{figure}
\caption{D2KSpipi NON Turbo}
\includegraphics[scale=0.3]{images/projection_plots/NO_TURBO/max_Signal_Eff.eps}
\end{figure}
\end{column}
\begin{column}{0.5\textwidth}
\begin{figure}
\caption{D2KSpipi Turbo}
\includegraphics[scale=0.3]{images/projection_plots/TURBO/max_Signal_Eff.eps}
\end{figure}
\end{column}
\end{columns}
\end{frame}

\begin{frame}{Projection Plots: MVA Signal Efficiency}
\topline
\begin{columns}
\begin{column}{0.5\textwidth}
\begin{figure}
\caption{D2KSpipi NON Turbo}
\includegraphics[scale=0.3]{images/projection_plots/NO_TURBO/MVA_Signal_Eff.eps}
\end{figure}
\end{column}
\begin{column}{0.5\textwidth}
\begin{figure}
\caption{D2KSpipi Turbo}
\includegraphics[scale=0.3]{images/projection_plots/TURBO/MVA_Signal_Eff.eps}
\end{figure}
\end{column}
\end{columns}
\end{frame}

\begin{frame}{Projection Plots: max MVA Signal Efficiency}
\topline
\begin{columns}
\begin{column}{0.5\textwidth}
\begin{figure}
\caption{D2KSpipi NON Turbo}
\includegraphics[scale=0.3]{images/projection_plots/NO_TURBO/max_MVA_Signal_Eff.eps}
\end{figure}
\end{column}
\begin{column}{0.5\textwidth}
\begin{figure}
\caption{D2KSpipi Turbo}
\includegraphics[scale=0.3]{images/projection_plots/TURBO/max_MVA_Signal_Eff.eps}
\end{figure}
\end{column}
\end{columns}
\end{frame}

\backupend

\end{document}
